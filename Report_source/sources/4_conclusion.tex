In conclusion, the analysis of the spectral lines data with Matlab and the cleaning up of the data of the 4 spectra has allowed for a more accurate determination of the peak locations of the spectral lines. By identifying the signal characteristics and finding the peaks to take away from the continuum curve, a polynomial function was able to be fit to flatten the curve, resulting in more precise peak measurements. Additionally, the fitting of Gaussian curves to the spectral lines provided more detailed information about their characteristics, which were then compared to the NIST database values. Through this process, it was determined that the spectral lines belonged to the element chromium. Overall, This study demonstrates the significance of meticulous data analysis in accurately characterizing spectral lines and identifying the elements that produce them.